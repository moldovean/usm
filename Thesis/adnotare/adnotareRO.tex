\documentclass[a4paper,12pt]{article}
\usepackage{hyperref}

\usepackage[utf8]{inputenc}
\title{\textbf{Applications of Homogeneous Markovian Processes in Economics and Latent Parameters Estimation for Discrete State Hidden Markov Models}}
\author{\textit{Student}: Adrian Vrabie \href{adrian@vrabie.net}{adrian@vrabie.net}\\
		\textit{Academic Adviser}: Dr. habil. Dmitrii Lozovanu, University Professor \\
\href{lozovanu@math.md}{lozovanu@math.md}}
\date{\today}

\begin{document}
\maketitle
\begin{abstract}
Procesele Marcoviene sunt ubicuitare (omniprezente) într-o gamă largă de aplicaţii şi are un grad înalt de explicaţie a datelor în pofida modelului matematic simplist. Un model mai general, Modelul Markovian Latent, nu este pe deplin valorificat din cauza dificultăţii estimării parametrilor acestuia. Această teză este dedicată cercetării algoritmilor care ar fi utili estimării matricei coeficienţilor proceselor Markoviene latente. Din păcate, această problemă nu are o soluţie analitică şi necesită aplicarea metodelor numerice. 
În prima parte sunt introduse procesele Markoviene simple si aplicaţiile acestora în Economie. Partea a doua prezintă metode şi algoritmi de tip EM (Expectation Maximization) pentru estimarea probabilităţilor de tranziţie între stările latente. 
Teza propune metode de eficientizare a implementării algoritmului Baum-Welch. 
Având matricea stocastică determinată, putem folosi metodele de programare dinamică propuse de Lozovanu şi Pickle pentru determinarea probabilităţilor limită. 


\end{abstract}

\end{document}
